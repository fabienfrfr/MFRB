\documentclass{article}
\usepackage{tikz}
\usetikzlibrary{arrows.meta, positioning, shapes.geometric}
\usepackage{amsmath} % Utile pour les maths

\begin{document}

\begin{figure}[ht]
\centering
\begin{tikzpicture}[
    firm/.style={rectangle, draw=blue!60, fill=blue!5, very thick, minimum size=1cm},
    param/.style={diamond, draw=red!60, fill=red!5, very thick, minimum size=1cm, aspect=1.5}, % Augmenter l'aspect pour les labels plus longs
    arrow/.style={->,>=Stealth,thick},
    label/.style={text width=3cm, align=center},
    decision/.style={circle, draw=black!60, fill=gray!10, minimum size=0.5cm} % Nouveau style pour a_i
]

% Firmes
\node[firm] (f1) at (0,0) {Firm 1};
\node[firm] (f2) at (3,0) {Firm 2};
\node[firm] (fN) at (7.5,0) {Firm $N$}; % Écart plus grand pour les points de suspension

% Points de suspension pour la série
\node at (5.25, 0) {$\dots$};

% Paramètres
\node[param] (gamma) at (3,2.2) {$\gamma$ (Competitive Pressure)};
\node[param] (k) at (3,-2.2) {$k$ (Automation Cost)};
\node[param] (beta) at (0,-2.2) {$\beta$ (Baseline Gain)};

% Flèches de compétition (gamma) - Les firmes sont liées par gamma
\draw[arrow, blue!60] (gamma) -- (f1);
\draw[arrow, blue!60] (gamma) -- (f2);
\draw[arrow, blue!60] (gamma) -- (fN);
% Suggestion d'interdépendance via gamma (plus conceptuel)
\draw[<->, dashed, blue!60] (f1) -- (f2); 
\draw[<->, dashed, blue!60] (f2) -- (fN);

% Flèches de coût (k)
\draw[arrow, red!60] (k) -- (f1);
\draw[arrow, red!60] (k) -- (f2);
\draw[arrow, red!60] (k) -- (fN);

% Flèches de gain (beta)
\draw[arrow, green!60] (beta) -- (f1);
\draw[arrow, green!60] (beta) -- (f2);
\draw[arrow, green!60] (beta) -- (fN);

% Décision d'automatisation (a_i)
\node[decision, below=0.5cm of f1] (a1) {$a_1$};
\node[decision, below=0.5cm of f2] (a2) {$a_2$};
\node[decision, below=0.5cm of fN] (aN) {$a_N$};
\draw[arrow] (f1) -- (a1);
\draw[arrow] (f2) -- (a2);
\draw[arrow] (fN) -- (aN);

% Équilibre
% Utiliser align=center dans la node pour la formule
\node[ellipse, draw=black!60, fill=black!5, very thick, minimum width=4cm, text width=5cm, align=center] (eq) at (3.75,-5) {
    Cournot-Nash Equilibrium: \\
    $$ \bar{a}^* = \frac{\gamma + \beta}{2k + \gamma} $$
};

% Flèches vers l'équilibre (représentant la détermination de l'équilibre)
\draw[arrow] (a1) -- (eq);
\draw[arrow] (a2) -- (eq);
\draw[arrow] (aN) -- (eq);

% Résultat : risque de displacement
\node[label, below=0.5cm of eq] (risk) {Labor Displacement Risk if $\bar{a}^* \approx 1$};

% Légende
\matrix[draw, below right, xshift=1cm, yshift=1cm] at (current bounding box.south west) {
    \node[firm,label=right:Firm]; \\
    \node[param,label=right:Parameter]; \\
    \node[draw=blue!60,circle,minimum size=0.5cm,label=right:Competitive Pressure ($\gamma$) Input]; \\
    \node[draw=red!60,circle,minimum size=0.5cm,label=right:Automation Cost ($k$) Input]; \\
    \node[draw=green!60,circle,minimum size=0.5cm,label=right:Baseline Gain ($\beta$) Input]; \\
};

\end{tikzpicture}
\caption{Conceptual diagram of the firm-level automation model. Firms compete by adjusting their automation levels $a_i$ based on competitive pressure ($\gamma$), costs ($k$), and baseline gains ($\beta$). The equilibrium $\bar{a}^*$ determines the risk of labor displacement.}
\label{fig:schema}
\end{figure}

\end{document}