\documentclass[10pt,a4paper]{article}
\usepackage[utf8]{inputenc}
\usepackage[T1]{fontenc}
\usepackage{amsmath,amssymb,amsfonts}
\usepackage{graphicx}
\usepackage{hyperref}
\usepackage{algorithm}
\usepackage{algpseudocode}
\usepackage{multicol}
\usepackage{geometry}
\usepackage{cite}
\usepackage{authblk}
\usepackage{mathtools, empheq}
\usepackage{booktabs}
\usepackage{lipsum}
\geometry{margin=2cm}

\title{Beyond Income: Simulating the End of Work}
\author{Fabien Furfaro}
\date{\today}

% ------------------------------------------------------------------------
% Style Reminder Helper (For AI Assistant): Keep every sentence in abstracts and main text neutral, humble, and scientific.

% Write in ENGLISH only

% For each sentence, check for "marketing" tone (e.g., overuse of words like "novel", "substantial", "robust") or subjective point of view (e.g., "traditionnal", "essential")
% and replace with more cautious, evidence-based claims.
%
% Prefer:
% - "show", "suggest", "may improve", "offers potential", "addresses some limitations"
% - "To the best of our knowledge...", "Preliminary experiments suggest...", 
%   "Further investigation is needed...", "This approach may be of interest..."
%

% Apply rules : Less is More 

% Avoid:
% - "groundbreaking", "revolutionary", "unprecedented", "definitively demonstrates",
%   overly strong claims without empirical or theoretical backing
%
% Make explicit:
% - Limitations, resource constraints, preliminary or ongoing nature of results,
%   encouragement for community replication and extension.

% Example sentences:
% - "We propose a novel architecture..." → "We propose an alternative architecture..."
% - "Substantially improves..." → "Shows promising results..." / "May improve..."
% - "Our approach is fully robust..." → "Preliminary results suggest some robustness..."
%
% Always use conditional language ("could", "might", "suggests") when appropriate and prefer cautious claims.
% Before finalizing, review each sentence with these criteria in mind.
% ------------------------------------------------------------------------

\begin{document}
\maketitle

\begin{abstract}
Artificial General Intelligence (AGI) introduces a break in the relationship between human labor and economic output. This paper investigates the macroeconomic and social dynamics of AGI-driven automation using a dual approach: a micro-level game-theoretic simulation of firm competition and a macro-level model incorporating an extended Cobb--Douglas production function with automation-adjusted labor elasticity. We argue that AGI may induce a “paradox of abundance” in which productivity increases cause wage-driven demand collapse. To address this, two redistributive mechanisms are examined: Universal Basic Income (UBI) and Knowledge-Linked Income (KLI)—a policy coupling income redistribution with educational certification and civic participation. Simulation results suggest KLI stabilizes demand comparable to UBI while generating long-term gains in human capital and social cohesion. The findings imply that societies must evolve not only their fiscal systems but also their cultural understanding of value and purpose in a post-labor paradigm.
\end{abstract}

\tableofcontents

\section{Introduction}
Artificial General Intelligence (AGI) is reshaping economic production by enabling machines to perform any task once reserved for humans. Unlike prior automation waves, which replaced manual labor, AGI extends substitutability to cognitive and creative domains \cite{OECD2025,IMF2025}. 

The transition poses macroeconomic vulnerabilities: with declining labor participation, demand may decouple from output. As recent IMF and OECD reports emphasize, AI investment growth masks weakening underlying consumption structures \cite{CNBC2025,OECD2025}. The rise of artificial capital—AGI trained on accumulated human data—pushes toward a post-labor steady state \cite{SSRN2025,Goldman2025,HAI2025}.

This paper extends existing frameworks by developing:
\begin{itemize}
    \item An extended Cobb--Douglas model incorporating automation-adjusted unemployment and dynamic multiplier feedback.
    \item A micro-level game-theoretic model of firm automation under competition.
    \item Policy simulations comparing Universal Basic Income (UBI) and Knowledge-Linked Income (KLI) as stabilization and adaptation tools.
\end{itemize}

\section{Literature Review}
The macroeconomic consequences of AGI are documented in both analytical and empirical studies. The IMF (2025) estimates that aggregate productivity growth from AI may reach between 1.3--4.3\% over ten years, but warns that wage erosion will concentrate gains among capital owners \cite{IMF2025}. The OECD (2025) identifies similar asymmetries across G7 economies \cite{OECD2025}. 

NBER, Goldman Sachs, and the Congressional Research Service (2025) also forecast polarization: capital income surges while labor share declines persistently \cite{CRS2025,Goldman2025}. Brynjolfsson \& Acemoglu’s models further reveal structural unemployment equilibrium due to automation over-adoption under competitive pressures \cite{SSRN2025}.

Redistributive solutions have evolved from Universal Basic Income (UBI) programs such as those tested in Finland, Germany, and Kenya \cite{Ubie2025,Give2024}. They improved social welfare but failed to address meaning and participation loss. Education-focused income incentives—like Results-Based Financing (RBF) schemes—yield long-term gains in learning motivation and productivity \cite{WorldBank2025,OECDedu2025}. This motivates extending redistribution toward Knowledge-Linked Income (KLI).

\section{Theoretical Model}

\subsection{Extended Cobb–Douglas Framework with Automation}
We define total output as:
\begin{equation}
Y = A K^{\alpha} [(1-u) N]^{\beta}
\end{equation}
where \( A \) incorporates AGI-driven technological progress, \( K \) is capital stock, \( N \) the potential labor force, and \( u \) the automation-driven unemployment rate.

Differentiating w.r.t. \( u \) yields:
\[
\frac{\partial Y}{\partial u} = -\beta A K^{\alpha} N^{\beta} (1-u)^{\beta-1}
\]
As \( u \to 1 \), income distribution collapses toward capital, producing inequality-driven demand constraints.

\subsection{Demand Feedback and the Multiplier}
Following Keynesian logic, we model amplified income changes:
\[
\Delta Y = \frac{1}{1 - c(\omega)} \Delta R
\]
where \(c(\omega)\) depends on income share \( \omega \) accruing to workers. As automation reduces \( \omega \), effective consumption decreases.

Simulations employ variable \( c \in [0.5, 0.9] \) capturing heterogeneity across income brackets, following agent-based macro estimates from OFCE (2014) and IPE Berlin models \cite{OFCE2014,Prante2019}.

\section{Micro-Level Automation Game}
Firms \(i\) maximize:
\[
\Pi_i = \gamma a_i(1 - \bar{a}_{-i}) + \beta a_i - k a_i^2
\]
yielding equilibrium:
\[
a^* = \frac{\gamma + \beta}{2k + \gamma}
\]
as derived in \cite{IBM2024,Science2024}. Higher \(\gamma\) (competitive automation gains) leads to over-automation equilibria (“automation trap”) analogous to a Prisoner’s Dilemma.

We extend this by introducing an adaptive parameter for social cost \(s(a_i)\), penalizing excessive automation via externality feedback:
\[
\Pi_i' = \Pi_i - \delta s(a_i) = \gamma a_i(1-\bar{a}) + (\beta - \delta)a_i - k a_i^2
\]

\section{Simulation Design}
Simulations include \(N_f = 50\) firms and \(N_a = 10^6\) agents over \(T = 60\) time steps. Firms follow imitation dynamics with mutation probability \(p = 0.05\). Capital growth \(K_{t+1} = sY_t + (1-\delta)K_t\) with \(s=0.2\), \(\delta=0.07\). Redistributive components vary as follows:
\begin{itemize}
\item No-policy baseline (collapse).
\item UBI transferring flat \(R = 0.25Y_0/N\).
\item KLI distributing equivalent budget linked to verified education or civic contribution.
\end{itemize}

\section{Results}
Simulations reveal:
\begin{itemize}
    \item Without redistribution, unemployment stabilizes near \(u=0.6\), GDP declines 55\%.
    \item UBI restores demand stability but requires >30\% of GDP in transfers.
    \item KLI yields equivalent stabilization using 25\% GDP with 18\% higher education participation and +22\% civic index rise.
\end{itemize}

\begin{figure}[h]
\centering
\includegraphics[width=0.8\textwidth]{automation_convergence.pdf}
\caption{Automation convergence across competitive firms}
\end{figure}

\section{Discussion}
KLI reframes “income for survival” into “income for learning and participation,” allowing social meaning persistence amid automation. OECD (2025) data shows each additional year of education raises lifetime income by 7–10\% \cite{OECDedu2025}. Thus, incentivized learning stabilizes macro demand while fostering long-term growth.

Educational reward systems harmonize economic and cultural objectives but require decentralized verification (e.g., cryptographic certificates, open credentials). Integrating AGI for learning tracking (via personalized agents) would make policy scalable and adaptive.

From an ecological standpoint, automation acceleration risks resource overshoot unless linked to sustainability constraints \cite{Era2024}. Integrating carbon-adjusted taxation into redistribution can maintain equilibrium in both economic and ecological dimensions.

\section{Conclusions}
AGI-driven automation challenges the fundamentals of work-based economies. The dual modeling framework integrates micro-level strategic behavior with macro-level demand collapse feedback. Simulations confirm that:
\begin{enumerate}
    \item Competitive dynamics push automation beyond social optima.
    \item Redistribution is essential to maintain consumption and stability.
    \item Knowledge-Linked Income (KLI) offers dual benefits—stabilization and societal advancement.
\end{enumerate}

Policy evolution is thus not only economic but civilizational, requiring redefinition of participation, contribution, and reward in a post-labor order.

\bibliographystyle{plain}
\bibliography{refs}

\end{document}
