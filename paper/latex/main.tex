\documentclass[10pt,a4paper]{article}
\usepackage[utf8]{inputenc}
\usepackage[T1]{fontenc}
\usepackage{amsmath,amssymb,amsfonts}
\usepackage{graphicx}
\usepackage{hyperref}
\usepackage{algorithm}
\usepackage{algpseudocode}
\usepackage{multicol}
\usepackage{geometry}
\usepackage{cite}
\usepackage{authblk}
\usepackage{mathtools, empheq}
\usepackage{booktabs}
\usepackage{lipsum}
\geometry{margin=2cm}
\title{Simulating the End of Work and Money: A Microsimulation of AGI-Driven Automation}
\author{Fabien Furfaro}


% ------------------------------------------------------------------------
% Style Reminder Helper (For AI Assistant): Keep every sentence in abstracts and main text neutral, humble, and scientific.
% Write in ENGLISH only
% For each sentence, check for "marketing" tone (e.g., overuse of words like "novel", "substantial", "robust") or subjective point of view (e.g., "traditionnal", "essential")
% and replace with more cautious, evidence-based claims.
%
% Prefer:
% - "show", "suggest", "may improve", "offers potential", "addresses some limitations"
% - "To the best of our knowledge...", "Preliminary experiments suggest...",
%   "Further investigation is needed...", "This approach may be of interest..."
%
% Apply rules : Less is More
% Avoid:
% - "groundbreaking", "revolutionary", "unprecedented", "definitively demonstrates",
%   overly strong claims without empirical or theoretical backing
%
% Make explicit:
% - Limitations, resource constraints, preliminary or ongoing nature of results,
%   encouragement for community replication and extension.
% Example sentences:
% - "We propose a novel architecture..." → "We propose an alternative architecture..."
% - "Substantially improves..." → "Shows promising results..." / "May improve..."
% - "Our approach is fully robust..." → "Preliminary results suggest some robustness..."
%
% Always use conditional language ("could", "might", "suggests") when appropriate and prefer cautious claims.
% Before finalizing, review each sentence with these criteria in mind.
% ------------------------------------------------------------------------
\begin{document}
\maketitle

\begin{abstract}
The emergence of Artificial General Intelligence (AGI) raises questions about the future role of human labor in economic systems. This paper presents a microsimulation model to explore how competitive firm behavior under AGI-driven automation may lead to a reduction in human labor demand. Using a game-theoretic framework inspired by the Prisoner's Dilemma, we simulate firm-level automation decisions with the profit function \(\Pi_i(a_i, \bar{a}_{-i}) = \gamma a_i (1 - \bar{a}_{-i}) + \beta a_i - k a_i^2\). Simulation results suggest that, under competitive pressure, automation levels may converge toward high substitution of human labor, particularly when competitive advantages (\(\gamma\)) are significant and automation costs (\(k\)) are low. These findings indicate that AGI-driven automation could challenge traditional wage-based economies, though the extent of this impact appears to depend on contextual factors such as cost structures and competitive dynamics \cite{IMF2025,OECD2025,Smirnov2025a}.
\end{abstract}

\section{Introduction}
Recent studies suggest that Artificial General Intelligence (AGI) could disrupt traditional labor markets by automating cognitive and physical tasks at near-zero marginal cost \cite{IMF2025,OECD2025}. This paper explores a microsimulation approach to model firm-level automation decisions, framed as a Prisoner's Dilemma. We use the profit function \(\Pi_i(a_i, \bar{a}_{-i}) = \gamma a_i (1 - \bar{a}_{-i}) + \beta a_i - k a_i^2\), where firms iteratively adjust their automation levels toward the Cournot-Nash equilibrium \(a_i^* = \frac{\gamma (1 - \bar{a}_{-i}) + \beta}{2k}\). While macroeconomic models and policy responses are acknowledged as relevant contextual factors, this study focuses specifically on the micro-level dynamics of competitive automation \cite{Smirnov2025a,Acemoglu2024}.

\section{Theoretical Model: The Automation Trap}

\subsection{Profit Function and Cournot-Nash Equilibrium}
Each firm \(i\) selects an automation level \(a_i \in [0,1]\) to maximize its profit:
\[
\Pi_i(a_i, \bar{a}_{-i}) = \gamma a_i (1 - \bar{a}_{-i}) + \beta a_i - k a_i^2
\]
where:
\begin{itemize}
    \item \(\gamma\) represents the competitive advantage from automating more than others \cite{Stiefenhofer2025},
    \item \(\beta\) is the baseline profit from automation \cite{BeforeIT2024},
    \item \(k\) denotes the cost of automation \cite{Acemoglu2024},
    \item \(\bar{a}_{-i}\) is the average automation level of other firms.
\end{itemize}

The Cournot-Nash equilibrium is derived by setting \(\frac{\partial \Pi_i}{\partial a_i} = 0\):
\[
a_i^* = \frac{\gamma (1 - \bar{a}_{-i}) + \beta}{2k}
\]

\subsection{The Automation Trap}
In the absence of collective coordination, firms iteratively adjust \(a_i\) toward \(a_i^*\). This dynamic may create a "race to the bottom": as \(\bar{a}_{-i} \to 1\), \(a_i^* \to 1\). Preliminary simulations suggest that this could result in high levels of automation, potentially rendering human labor redundant under certain conditions. However, further investigation is needed to assess the robustness of this outcome across different parameter settings \cite{Smirnov2025a,Acemoglu2024}.

\section{Simulation Design and Results}

\subsection{Methodology}
We simulate \(N=10\) firms over \(T=1000\) rounds, with parameters \(\gamma \in \{0.5, 1.0, 2.0, 3.0\}\), \(k \in \{0.2, 0.8, 1.4, 2.0\}\), \(\beta=1.0\), and a mutation rate of 5\% to reflect innovation and avoid trivial equilibria. At each round, firms update \(a_i\) toward \(a_i^*\), reflecting competitive pressure.

\subsection{Simulation Outputs}
The simulation results are presented in Figure~\ref{fig:combined_results}, combining a heatmap of final automation levels and the dynamics of average automation over time. The heatmap values (Table~\ref{tab:heatmap_values}) show the mean automation levels after \(T=1000\) rounds, averaged over \(10\) runs for each parameter combination (\(\gamma\), \(k\)).

\begin{table}[h]
\centering
\caption{Final Automation Levels (mean over 10 runs)}
\label{tab:heatmap_values}
\begin{tabular}{lcccc}
\toprule
\(k \backslash \gamma\) & 0.5 & 1.0 & 2.0 & 3.0 \\
\midrule
0.2 & 0.907 & 0.893 & 0.924 & 0.920 \\
0.8 & 0.782 & 0.844 & 0.877 & 0.902 \\
1.4 & 0.453 & 0.469 & 0.644 & 0.734 \\
2.0 & 0.344 & 0.433 & 0.545 & 0.566 \\
\bottomrule
\end{tabular}
\end{table}


\begin{figure}[h]
\centering
\includegraphics[width=0.95\textwidth]{combined_automation_results.pdf}
\caption{Simulation results: (left) Heatmap of final automation levels for varying \(\gamma\) (competitive advantage) and \(k\) (cost of automation); (right) Dynamics of average automation (mean \(\pm\) standard deviation) over 1000 rounds.}
\label{fig:combined_results}
\end{figure}

\subsection{Key Observations}
\begin{itemize}
    \item \textbf{High Competitive Advantage (\(\gamma = 2.0, 3.0\))}: Automation levels converge toward \(a_i \approx 0.9\) even for moderate costs (\(k = 0.2, 0.8\)), suggesting that firms prioritize automation to gain a competitive edge. This aligns with theoretical predictions of an "automation trap," where competitive pressure drives firms toward high automation despite potential long-term demand collapse \cite{Acemoglu2024,Smirnov2025a}.

    \item \textbf{Moderate Competitive Advantage (\(\gamma = 1.0\))}: Automation levels are sensitive to cost (\(k\)). For \(k = 2.0\), automation remains low (\(a_i \approx 0.43\)), indicating that firms limit automation when costs outweigh competitive gains. This suggests a potential equilibrium where automation is partial, preserving some human labor \cite{Stiefenhofer2025}.

    \item \textbf{Low Competitive Advantage (\(\gamma = 0.5\))}: Automation levels are consistently low across all costs (\(a_i \leq 0.91\)), implying that firms see little incentive to automate aggressively without strong competitive pressure \cite{BeforeIT2024}.

    \item \textbf{Cost Sensitivity}: For all \(\gamma\), automation decreases as \(k\) increases (e.g., \(a_i\) drops from \(\approx 0.9\) to \(\approx 0.34\) as \(k\) increases from 0.2 to 2.0 for \(\gamma = 0.5\)). This highlights the role of cost as a natural brake on automation, even under competitive pressure \cite{OECD2025}.
\end{itemize}

\subsection{Implications: The End of Work and Money}
The results suggest that:
\begin{itemize}
    \item \textbf{The End of Work}: Under high competitive advantage (\(\gamma \geq 2.0\)), firms automate aggressively (\(a_i \geq 0.87\)), potentially rendering human labor redundant in certain sectors. This supports predictions that AGI could replace many human tasks by 2040, though the extent depends critically on automation costs (\(k\)) and competitive dynamics \cite{Smirnov2025a,IMF2025}.

    \item \textbf{Conditional Obsolescence of Money}: If automation reaches \(a_i \approx 1\) (as observed for \(\gamma = 3.0, k = 0.2\)), wage-based demand may collapse, challenging the role of money as a medium of exchange. However, for \(\gamma \leq 1.0\) or high \(k\), partial automation (\(a_i < 0.7\)) suggests that human labor—and thus wage-based economies—may persist in certain contexts \cite{Smirnov2025b,Stiefenhofer2025}.
\end{itemize}

\section{Discussion}

\subsection{Interpretation of Results}
The simulation results provide preliminary evidence that competitive AGI-driven automation may lead to a Prisoner's Dilemma-like outcome, where rational firm behavior results in high automation levels under specific conditions. Key observations include:

\begin{itemize}
    \item \textbf{Competitive Pressure Dominates for \(\gamma \geq 2.0\)}: Firms automate aggressively even at moderate costs, suggesting that competitive dynamics alone may suffice to drive labor obsolescence in some sectors. This aligns with theoretical predictions of an "automation trap," where firms cannot unilaterally reduce automation without losing market share \cite{Acemoglu2024,Smirnov2025a}.

    \item \textbf{Cost as a Mitigating Factor}: High automation costs (\(k \geq 1.4\)) limit automation even for \(\gamma = 3.0\), implying that policy tools (e.g., taxation on automation) could potentially slow the transition to full automation. This contrasts with deterministic predictions of labor obsolescence and suggests that economic outcomes depend on controllable parameters \cite{OECD2025,Stiefenhofer2025}.

    \item \textbf{Heterogeneity in Automation}: The variation in final automation levels (0.34 to 0.92) across parameter combinations underscores the importance of context. AGI's impact on labor may not be uniform across industries or regions, depending on local competitive pressures and cost structures \cite{BeforeIT2024,Smirnov2025b}.
\end{itemize}

\subsection{Limitations and Further Research}
This study has several limitations:
\begin{itemize}
    \item \textbf{Simplified Firm Behavior}: The model assumes homogeneous firms with perfect information, abstracting away real-world complexities such as heterogeneous productivity, regulatory constraints, or strategic alliances \cite{Acemoglu2024}.

    \item \textbf{Static Parameters}: \(\gamma\), \(\beta\), and \(k\) are fixed, yet real-world automation costs and competitive advantages evolve over time. Dynamic parameterization could yield more realistic trajectories \cite{Smirnov2025a}.

    \item \textbf{No Macroeconomic Feedback}: The simulation does not model demand collapse or policy responses (e.g., UBI), which may alter automation incentives. Future work should integrate these feedback loops \cite{OECD2025,Stiefenhofer2025}.
\end{itemize}

\subsection{Broader Implications}
The results suggest that AGI-driven automation is \textbf{context-dependent}. While high competitive pressure (\(\gamma\)) and low costs (\(k\)) may lead to labor obsolescence in some sectors, policy interventions or technological constraints could mitigate this outcome. This challenges deterministic narratives about the "end of work" and highlights the need for proactive governance to shape AGI's economic impact \cite{IMF2025,Smirnov2025a}. Further research should explore:
\begin{itemize}
    \item The interaction between automation and demand collapse \cite{OECD2025},
    \item Policy tools to align firm incentives with social welfare \cite{Stiefenhofer2025},
    \item Sectoral heterogeneity in automation adoption \cite{BeforeIT2024}.
\end{itemize}

\section{Conclusion}
AGI-driven automation, when modeled as a competitive process, may lead to high levels of automation and potential labor obsolescence under certain conditions. The simulation supports the hypothesis that AGI could challenge traditional wage-based economies, though the extent of this impact depends on competitive dynamics and cost structures. These findings suggest that the future of work and money in an AGI-driven economy is not predetermined but may be shaped by policy choices and technological constraints. Further investigation is needed to assess the generalizability of these findings and explore potential policy interventions \cite{IMF2025,OECD2025,Smirnov2025a,Smirnov2025b,Acemoglu2024,Stiefenhofer2025,BeforeIT2024}.

\bibliographystyle{plain}
\bibliography{refs}
\end{document}
