\documentclass[10pt,a4paper]{article}
\usepackage[utf8]{inputenc}
\usepackage[T1]{fontenc}
\usepackage{amsmath,amssymb,amsfonts}
\usepackage{graphicx}
\usepackage{hyperref}
\usepackage{algorithm}
\usepackage{algpseudocode}
\usepackage{booktabs}
\usepackage{lipsum}
\usepackage{geometry}
\usepackage{cite}
\usepackage{authblk}
\usepackage{mathtools, empheq}
\usepackage{multicol}
\geometry{margin=2cm}
\title{Simulating the End of Work and Money: A Microsimulation of AGI-Driven Automation}
\author{\large Fabien Furfaro\thanks{\texttt{fabien.furfaro@gmail.com}}}
\date{\large 2025}

\begin{document}
\maketitle

\begin{abstract}
The development of Artificial General Intelligence (AGI) raises questions about the future of labor and monetary systems. This paper models firm-level automation decisions as a non-cooperative game, structurally analogous to the Prisoner's Dilemma. Simulations show that automation levels depend critically on competitive pressure (\(\gamma\)) and implementation costs (\(k\)), with high \(\gamma/k\) ratios leading to near-complete labor substitution. These results suggest that policy tools—such as automation taxation or universal basic income funded by AI rents—could mitigate adverse economic outcomes. The study underscores that AGI's impact on work and money is not technologically predetermined but shaped by institutional choices.
\end{abstract}

\section{Introduction}
\subsection{Context: AGI and the Future of Labor}
% --- Contenu de la v1 (2 premiers paragraphes) ---
\lipsum[1] % Remplacer par ton texte : enjeux AGI, substitution des tâches, débats sur le UBI.

\subsection{Research Problem and Contribution}
% --- Nouveau : problématique explicite + contribution en 3 points ---
Our study addresses the following question: \textit{Under what conditions do competitive dynamics between firms lead to excessive automation, and how do these outcomes challenge traditional economic equilibria?} We contribute to the literature by:
\begin{itemize}
    \item Formalizing firm-level automation decisions as a \textbf{non-cooperative game} with a Prisoner's Dilemma structure.
    \item Simulating the impact of \textbf{competitive pressure (\(\gamma\))} and \textbf{automation costs (\(k\))} on long-term equilibrium outcomes.
    \item Discussing policy implications, including \textbf{automation taxation} and \textbf{universal basic income (UBI)} funded by AI rents.
\end{itemize}

\subsection{Outline}
% --- Nouveau : 1 phrase par section ---
Section~\ref{sec:theory} presents the theoretical framework. Section~\ref{sec:model} describes the simulation methodology and results. Section~\ref{sec:implications} discusses the implications for labor markets and policy design. Section~\ref{sec:conclusion} concludes and outlines future research directions.

\section{Theoretical Framework: Why a Non-Cooperative Game?}
\label{sec:theory}
\subsection{Limitations of Existing Approaches}
% --- Contenu de la v1 (comparaison avec CES, réseaux, options réelles) ---
\lipsum[2] % Remplacer par : limites des modèles CES (pas de stratégie), réseaux (complexité), options réelles (dynamique seulement).

% --- Nouveau : Tableau comparatif ---
\begin{table}[ht]
\centering
\caption{Comparison of Modeling Approaches for AGI-Driven Automation}
\label{tab:approaches}
\begin{tabular}{@{}llll@{}}
\toprule
\textbf{Approach}       & \textbf{Strengths}                          & \textbf{Limitations}                     & \textbf{Relevance to AGI} \\
\midrule
CES Production Functions & Simple, macro-level insights                & No strategic firm behavior               & Low                        \\
Network Models           & Captures interdependencies                 & High complexity, data-intensive         & Medium                     \\
Real Options             & Handles uncertainty and irreversibility    & Static competitive environment           & Medium                     \\
\textbf{Our Game-Theoretic Model} & \textbf{Firm-level strategies, clear policy levers} & \textbf{Homogeneous firms, static} & \textbf{High} \\
\bottomrule
\end{tabular}
\end{table}

\subsection{Profit Function and Cournot-Nash Equilibrium}
% --- Contenu de la v1 (équation de profit, équilibre symétrique) ---
The profit function for firm \(i\) is:
\[
\Pi_i(a_i, \bar{a}_{-i}) = \gamma a_i (1 - \bar{a}_{-i}) + \beta a_i - k a_i^2,
\]
where \(\gamma\) is the competitive advantage, \(k\) the cost of automation, and \(\beta\) the baseline gain. The symmetric Cournot-Nash equilibrium is:
\[
\bar{a}^* = \min\left(1, \frac{\gamma + \beta}{2k + \gamma}\right).
\]

% --- Nouveau : Exemple numérique ---
For \(\gamma = 2\), \(k = 0.5\), and \(\beta = 1\), the equilibrium \(\bar{a}^* = 0.8\) implies that firms automate 80\% of tasks, even if social costs are high.

\subsection{Assumptions and Scope}
% --- Nouveau : Justification des hypothèses ---
We assume homogeneous firms to isolate the effect of \(\gamma\) and \(k\), but discuss extensions with heterogeneity in Section~\ref{sec:implications}.

\section{Model and Simulation Results}
\label{sec:model}
\subsection{Methodology}
% --- Contenu de la v1 (algorithme d'imitation-mutation) ---
Firms update their automation levels \(a_i\) via:
\begin{enumerate}
    \item \textbf{Imitation}: Adopt the level of a more profitable firm with probability \(p_{\text{adopt}} = \frac{1}{1 + e^{-(\Pi_j - \Pi_i)}}\).
    \item \textbf{Mutation}: Random perturbation with 5\% probability.
\end{enumerate}

% --- Nouveau : Tableau des paramètres ---
\begin{table}[ht]
\centering
\caption{Simulation Parameters and Economic Interpretation}
\label{tab:params}
\begin{tabular}{@{}lll@{}}
\toprule
\textbf{Parameter} & \textbf{Values}       & \textbf{Interpretation}               \\
\midrule
\(\gamma\)         & 0.5, 1.0, 2.0, 3.0    & Competitive advantage (low to high)   \\
\(k\)              & 0.2, 0.8, 1.4, 2.0    & Automation cost (low to high)         \\
\(\beta\)          & 1.0                   & Baseline automation gain              \\
\bottomrule
\end{tabular}
\end{table}

\subsection{Results}
% --- Contenu de la v1 (heatmap, dynamiques temporelles) ---
Figure~\ref{fig:heatmap} shows final automation levels \(\bar{a}^*\) for varying \(\gamma\) and \(k\). Figure~\ref{fig:dynamics} compares simulation and analytical convergence.

% --- Nouveau : Tableau des résultats ---
\begin{table}[ht]
\centering
\caption{Equilibrium Automation Levels \(\bar{a}^*\) for Varying \(\gamma\) and \(k\)}
\label{tab:results}
\begin{tabular}{@{}cccc@{}}
\toprule
\(\gamma \downarrow\), \(k \rightarrow\) & 0.2  & 0.8  & 1.4  & 2.0  \\
\midrule
0.5                                 & 0.53 & 0.21 & 0.15 & 0.12 \\
1.0                                 & 0.76 & 0.45 & 0.30 & 0.23 \\
2.0                                 & 0.96 & 0.83 & 0.63 & 0.50 \\
3.0                                 & 1.00 & 0.88 & 0.70 & 0.57 \\
\bottomrule
\end{tabular}
\end{table}

% --- Figures ---
\begin{figure}[ht]
\centering
\includegraphics[width=0.45\textwidth]{schema_modele.pdf}
\caption{Conceptual diagram of the model: firms, competitive pressure (\(\gamma\)), automation costs (\(k\)), and equilibrium \(\bar{a}^*\).}
\label{fig:schema}
\end{figure}

\begin{figure}[ht]
\centering
\includegraphics[width=0.9\textwidth]{combined_automation_results.pdf}
\caption{(Left) Heatmap of final automation levels \(\bar{a}^*\). (Right) Dynamics of average automation (simulation vs. analytical).}
\label{fig:results}
\end{figure}

\section{Implications: Toward a Post-Labor Economy?}
\label{sec:implications}
\subsection{Key Scenarios}
% --- Nouveau : 3 scénarios ---
\begin{itemize}
    \item \textbf{Full automation} (\(\gamma = 3, k = 0.2\)): \(\bar{a}^* \approx 1\) → Risk of demand collapse.
    \item \textbf{Partial automation} (\(\gamma = 1, k = 1.4\)): \(\bar{a}^* \approx 0.3\) → Human labor persists.
    \item \textbf{Unstable equilibrium} (\(\gamma = 2, k = 0.8\)): \(\bar{a}^* \approx 0.8\) → Sensitive to shocks.
\end{itemize}

\subsection{Policy Responses}
% --- Contenu de la v1 (UBI, taxation) + encadré ---
\begin{figure}[ht]
\centering
\includegraphics[width=0.7\textwidth]{scenarios_policies.pdf}
\caption{Three automation scenarios and associated policy responses.}
\label{fig:scenarios}
\end{figure}

\subsection{Limitations and Extensions}
% --- Contenu de la v1 (limites) + nouveau ---
\begin{itemize}
    \item \textbf{Assumptions}: Homogeneous firms, fixed \(\beta\).
    \item \textbf{Future work}: Heterogeneity, macro feedback, dynamic \(k(t)\).
\end{itemize}

\section{Conclusion}
\label{sec:conclusion}
% --- Contenu de la v1 (synthèse) + perspectives ---
Our study shows that:
\begin{itemize}
    \item AGI-driven automation is \textbf{not inevitable} but depends on \(\gamma\) and \(k\).
    \item Competitive dynamics can lead to \textbf{excessive automation}, even if socially suboptimal.
    \item Policies like \textbf{automation taxation} or \textbf{UBI} could mitigate risks.
\end{itemize}
Future research should:
\begin{enumerate}
    \item Calibrate \(\gamma\) and \(k\) using sectoral data.
    \item Integrate macroeconomic feedback loops.
    \item Test policy interventions via simulation.
\end{enumerate}

\bibliographystyle{plain}
\bibliography{refs}

\end{document}
